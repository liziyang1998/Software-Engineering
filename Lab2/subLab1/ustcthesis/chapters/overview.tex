\chapter{总体概述}

Describes the general elements that may affect the product and the requirements on the product. It includes the following four parts. Note that this section should not describe the specific requirements, instead, it makes the specific requirements to be described more understandable.

本节描述影响产品和产品需求的一般因素。由以下4个部分构成。 有一点需说明的是本节不描述具体的需求,只是使那些将要描述的具体需求更易于理解。
\section{软件概述}
\subsection{项目介绍}
%Describe the context and origin of the project being specified in this SRS. For example, state whether this project is a follow-on member of a project family, a replacement for certain existing systems, or a new, self-contained project.

%描述本软件需求所描述的项目的背景。例如:本项目是一系列版本中的一个,或者是替代某个已经存在的系统,还是一个新的独立的项目。

This project is used to replace the comprehensive educational system of colleges and universities. In view of the problems of single interface and disorder of the old educational system, it is proposed to develop a new comprehensive educational system to replace the old educational system.

本项目用于代替高校的综合教务系统,鉴于旧版教务系统存在界面单一,功能杂乱等问题,拟开发一个新的综合教务系统来代替旧版教务系统


\subsection{产品环境介绍}
The integrated educational administration system defined in the software requirements specification is a component of the university management system and should have the following attention:
\begin{itemize}
\item The large system is a university management system, and has a comprehensive educational system defined by this document and a comprehensive information system for schools.
\item The interface of the software includes: student electives, student status information, exam information, transfer professional application, teaching assistant management system, teacher management system
\item Hardware platform:
\begin{itemize}
\item Client: **
\item CPU: **
\item RAM: **
\item Resolution: **
\item Service-Terminal: **
\item CPU: **
\item RAM: **
\item Hard Disk: **
\end{itemize}
\item Operating system and version : UNIX/LINUX/Windows2000 or above
\item Database version : Microsoft SQL Server 2017 、Jdk1.6 and above
\end{itemize}

%It is very helpful to describe the main components, interconnection and external interfaces of the larger system/project by Block Diagram. This part should not provide a detailed design solution, or detailed design constraint for the solution (the detailed design constraint will be described in the section of specific requirement). This section is the basis of the design constraints.

该软件需求说明书中定义的综合教务系统是高校管理系统中的组件,应有以下注意:

	A. 该大系统为高校管理系统,有本文档定义的综合教务系统、学校综合信息系统等。

	B. 本软件的接口有:学生选课、学籍信息、考试信息、转专业申请、助教管理系统、教师管理系统
    	
	C. 硬件平台:

	客户端: **

	CPU: **

	内存: **

	分辨率: **

	服务器端: **

	CPU: **

	内存: **

	硬盘: **

	D. 操作系统和版本 UNIX/LINUX/Windows2000或以上版本

	E. 数据库版本 Microsoft SQL Server 2017  Jdk1.6及以上


\section{软件功能}
The functions that the integrated educational system needs to achieve are:

Course selection / withdrawal / class schedule inquiry: Students can choose and withdraw classes before the start of the new semester or two weeks before the start of the new semester, and can conduct class schedule inquiry throughout the semester.

Course Lottery: For classes with more students than the number of courses, draw lots and select the upper limit of the course from all the candidates.

Job Release/Teaching Resource Sharing: Teachers or TAs publish assignments and teaching resources that students can browse and download online

Course Notice Announcement/Email: The teacher or teaching assistant will post the course notice and email, and the student can learn through the network.

Grade Enquiry: Students can check their grades after the teacher releases the grades.

Assessment system: Students objectively evaluate the courses and teachers of the semester after the course is over.

该综合教务系统需实现的功能有:

课程设置:教学秘书在学期开始前设置本学院学生应该上的课和教师所开设的课

选课/退课/课表查询:学生在新学期开始前或新学期开始前两周进行选课和退课操作,并在整个学期可以进行课表查询

课程抽签:对于选课人数大于课程设置人数的课,进行抽签,从所有选课人里面等概率选取课程上限人数

作业发布/教学资源共享:教师或助教发布作业和教学资源,学生可以在线浏览和下载

课程通知公告/邮件:教师或助教发布课程通知公告和邮件,学生可以通过网络了解

成绩查询:学生可以在教师发布成绩后查询自己的成绩

评课系统:学生在课程结束后,客观的评价本学期的课程和教师

\section{用户特征}

Users are divided into three categories, the first is a teacher, the second is a teaching assistant, and the third is a student.

Teachers and teaching assistants need to be proficient in job posting, teaching resource release, course notice announcement, mail and grade registration and release.

Students should be proficient in using electives, withdrawals, class schedule enquiries, grade enquiries and assessments.

System operators are divided into two categories, the first is system maintenance, and the second is teaching secretary.

System maintenance should maintain the normal operation of the system, especially during the peak period of class withdrawal

Teaching secretary should be proficient in applications such as curriculum settings


用户分为三类,第一类是教师,第二类是助教,第三类是学生。
\begin{itemize}
\item 教师、助教需要熟练使用作业发布、教学资源发布、课程通知公告、邮件和成绩的登记和发布
\item 学生应该熟练使用选课、退课、课表查询、成绩查询和评课
\end{itemize}

系统操作者分为两类,第一类是系统维护,第二类是教学秘书
\begin{itemize}
\item 系统维护应该维护系统的正常运转,尤其在选课退课高峰期
\item 教学秘书应该熟练使用课程设置等应用
\end{itemize}


\section{假设和依赖关系}

The language used is Java SE 12

Operating system and version UNIX/LINUX/Windows2000 or above

Database version Microsoft SQL Server 2017 Jdk1.6 and above

使用语言为Java SE 12

操作系统和版本 UNIX/LINUX/Windows2000或以上版本

数据库版本 Microsoft SQL Server 2017  Jdk1.6及以上

