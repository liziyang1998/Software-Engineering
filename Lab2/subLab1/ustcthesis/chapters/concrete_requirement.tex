\chapter{具体需求}
\section{功能需求}
This section describes how the input of the software is translated to the output. It describes the essential action the software must perform.

本子章节应描述软件产品的输入怎样被转换成输出。它描述了软件必须执行的基本动作。 

\subsection{R.INTF.CALC.001 课程设置}
\subsubsection{介绍}

The course is managed by the teaching secretary.

Includes opening a new course, deleting a course, setting a maximum number of courses, adding a course description, and adding a teacher contact

For compulsory courses, set the class and class time

For elective courses, set the time and place of class

Course descriptions include but are not limited to: credits, hours, experimental hours, course number

For illegal input, incorrect input, invalid input, you should refuse to fill in the database.

课程设置由教学秘书管理。

包含开设新课程,删除课程,设置课程人数上限,添加课程描述,添加教师信息

对于必修课程,设置选课班级和上课时间地点

对于选修课程,设置上课时间地点

课程描述包括但不限于:学分、学时、实验学时、课程编号

对于非法输入、错误输入、无效输入应该拒绝填写进入数据库


\subsubsection{输入}
A detailed description of all input data for this function, including:

Input source: manual keyboard input

Quantity: several

Time requirement: before the start of the semester

Valid input range with precision and tolerance: input should be an integer, 32-bit int

对该功能所有输入数据的详细描述,包括:

		输入来源:人工键盘输入

		数量:若干

		时间要求:学期开始前

		包含精度和容忍度的有效输入范围:输入应为整数,为32位int型		

\subsubsection{处理}
This subsection should describe all the operations performed on the input data and how to obtain the output. This includes the following specifications:

A. Validation of input data.

B. The exact order of operations: set new lessons -> (set class selection) -> set class time and place -> set the maximum number of courses -> add course description -> add teacher information

C. Response to anomalies, such as:

Overflow: Reject write

Communication failed: write refused

Error handling: prompt specific error message

D. Any method (such as equations, mathematical algorithms, logical operations) used to convert system inputs to corresponding outputs:
Directly displayed on the client

E. Validation of the output data.

本子段落应描述对输入数据所执行的所有操作和如何获得输出的过程。这包括下列规格:

A. 输入数据的有效性检测。

B. 操作的确切次序:设置新课->(设置选课班级)->设置上课时间地点->设置课程人数上限->添加课程描述->添加教师信息

C. 对异常情况的回应,例如:
		
	溢出:拒绝写入

	通信失败:拒绝写入
	
	错误处理:提示具体错误信息
	
D. 用于把系统输入转换到相应输出的任何方法(诸如方程式,数学算法,逻辑操作):
	直接显示在客户端
		
E.	对输出数据的有效性检测。
\subsubsection{输出}
A detailed description of all output data for this function, this description includes:

Output to the client

Quantity: several

Effective output range with precision and tolerance: integer, 32-bit int

Handling illegal values: return error code

Error message: return error message

对该功能所有输出数据的详细描述,这个描述包括:
		
	输出的到客户端

	数量:若干

	包含精确度和容忍度的有效输出范围:整数,32位int型

	对非法值的处理:返回错误代码
	
	错误消息:返回错误信息

\subsection{R.INTF.CALC.002 选课/退课/课表查询}
\subsubsection{介绍}

Course selection, withdrawal, and class schedule enquiries are completed by students.

Select electives in the system at the specified course time, review the compulsory courses that have been scheduled, and change shifts as needed.

Withdrawal at the specified course time or personalization at the time of class selection

View schedule information throughout the semester

选课、退课、课表查询由学生完成。

在规定的选课时间在系统中选择选修课,查看已经安排好的必修课并根据需求换班。

在规定的选课时间退课或不在选课时间个性化退课

整个学期内可以查看课表信息

\subsubsection{输入}
A detailed description of all input data for this function, including:

Input source: manual input

Quantity: several

Time requirement: at the specified course time, usually before the start of the semester

Valid input range with precision and tolerance: input should be an integer, 32-bit int

对该功能所有输入数据的详细描述,包括:

		输入来源:人工输入

		数量:若干

		时间要求:在规定的选课时间,一般为学期开始前

		
\subsubsection{处理}
A. Validation of input data.

B. The exact order of the operation: the student chooses an elective course and is allowed to withdraw only when there is the course and is an elective course.

C. Response to anomalies, such as:

Overflow: Reject write

Communication failed: write refused

Error handling: prompt specific error message

D. Any method (such as equations, mathematical algorithms, logical operations) used to convert system inputs to corresponding outputs:

Direct output

E. Validation of the output data.

A. 输入数据的有效性检测。

B. 操作的确切次序:学生选择选修课,只有在有该门课并且为选修课的时候允许退课

C. 对异常情况的回应,例如:

	溢出:拒绝写入

	通信失败:拒绝写入

	错误处理:提示具体错误信息

D. 用于把系统输入转换到相应输出的任何方法(诸如方程式,数学算法,逻辑操作):

	直接输出
		
E.	对输出数据的有效性检测。
\subsubsection{输出}
A detailed description of all output data for this function:

Where is the output (eg printer, file): client

Quantity: several

Time requirement: at the specified course time, usually before the start of the semester

Effective output range with precision and tolerance: integer, 32-bit int

Handling illegal values: return error code

Error message: return error message

对该功能所有输出数据的详细描述:

	输出的到何处(如打印机,文件):客户端

	数量:若干

	时间要求:在规定的选课时间,一般为学期开始前

	对非法值的处理:返回错误代码

	错误消息:返回错误信息
	

\subsection{R.INTF.CALC.003 课程抽签}
\subsubsection{介绍}
The course lottery is applied to the case where the number of enrolled electives is greater than the upper limit of the course. If the number of compulsory courses for a teacher reaches the upper limit, the student is no longer allowed to choose the teacher's course.

Course lottery should be done automatically by the system.

课程抽签应用于选修课报名人数大于课程上限的情况。如果某老师的必修课人数达到上限,则不再允许学生选该老师的该门课。

课程抽签应该由系统自动完成。

\subsubsection{输入}
A detailed description of all input data for this function, including:

Input source: student information

Quantity: several

Time requirement: after the class is over

Valid input range with precision and tolerance: input should be an integer, 32-bit int

对该功能所有输入数据的详细描述,包括:

	输入来源:选课学生信息

	数量:若干

	时间要求:选课结束后

	包含精度和容忍度的有效输入范围:输入应为整数,为32位int型
		
\subsubsection{处理}

A. Validation of input data.

B. Only when the number of electives is greater than the maximum number of students in the elective course, the lottery will be activated and the lottery will be drawn.

C. Response to anomalies, such as:

Overflow: Reject write

Communication failed: write refused

Error handling: prompt specific error message

D. Any method (such as equations, mathematical algorithms, logical operations) used to convert system inputs to corresponding outputs:

Direct output

E. Validation of the output data.

A. 输入数据的有效性检测。

B. 只有在选修课选修人数大于课程上限人数的时候激活课程抽签,抽签为随即抽签。

C. 对异常情况的回应,例如:

	溢出:拒绝写入

	通信失败:拒绝写入

	错误处理:提示具体错误信息

D. 用于把系统输入转换到相应输出的任何方法(诸如方程式,数学算法,逻辑操作):

	直接输出选中学生信息

E.	对输出数据的有效性检测。
\subsubsection{输出}
A detailed description of all output data for this function:

Where is the output (eg printer, file): client

Quantity: several

Time requirement: after the lottery is over

Effective output range with precision and tolerance: integer, 32-bit int

Handling illegal values: return error code

Error message: return error message

对该功能所有输出数据的详细描述:

	输出的到何处(如打印机,文件):客户端

	数量:若干

	时间要求:抽签结束后

	包含精确度和容忍度的有效输出范围:整数,32位int型

	对非法值的处理:返回错误代码

	错误消息:返回错误信息
		

\subsection{R.INTF.CALC.004 作业发布/教学资源共享}
\subsubsection{介绍}
This function is used by teachers and teaching assistants to publish assignments and teaching resources for students to view and download.

该功能由教师和助教使用,发布作业和教学资源,方便学生查阅和下载

\subsubsection{输入}
A detailed description of all input data for this function, including:

Input source: manual input

Quantity: several

Time requirement: the semester is in progress

对该功能所有输入数据的详细描述,包括:

	输入来源:人工输入

	数量:若干

	时间要求:学期进行中
		
\subsubsection{处理}

A. Validation of input data.

B. The exact sequence of operations: Teachers or teaching assistants post assignments and teaching resources after the course, which students can review and download.

C. Response to anomalies, such as:

Overflow: refuse to write,

Communication failed: write refused

Error handling: prompt specific error message

D. Any method (such as equations, mathematical algorithms, logical operations) used to convert system inputs to corresponding outputs. For example, this might describe the following:

Direct output

E. Validation of the output data.

A. 输入数据的有效性检测。

B. 操作的确切次序:教师或助教在课程后发布作业和教学资源,学生可以查阅和下载。

C. 对异常情况的回应,例如:

	溢出:拒绝写入、

	通信失败:拒绝写入
	
	错误处理:提示具体错误信息

D. 用于把系统输入转换到相应输出的任何方法(诸如方程式,数学算法,逻辑操作)。例如,这可能描述下列方面:

	直接输出

E.	对输出数据的有效性检测。
\subsubsection{输出}

A detailed description of all output data for this function:

Where is the output (eg printer, file): client

Quantity: several

Time requirement: the semester is in progress

Handling illegal values: return error code

Error message: return error message

对该功能所有输出数据的详细描述:

	输出的到何处(如打印机,文件):客户端
	
	数量:若干
	
	时间要求:学期进行中
	
	对非法值的处理:返回错误代码
	
	错误消息:返回错误信息

\subsection{R.INTF.CALC.005 课程通知公告/邮件}
\subsubsection{介绍}

Relevant course information and announcements are issued by teachers or teaching assistants, and emails can be sent for reminders when necessary.

由教师或助教发布相关课程信息和公告,必要时候可以发送邮件进行提醒。

\subsubsection{输入}
A detailed description of all input data for this function, including:

Input source: manual input

Quantity: several

Time requirement: the semester is in progress

对该功能所有输入数据的详细描述,包括:

	输入来源:人工输入

	数量:若干

	时间要求:学期进行中
	
\subsubsection{处理}
A. Validation of input data.

B. The exact order of the operation: the teacher or the teaching assistant will post an announcement or email before the student can view it.

C. Response to anomalies, such as:

Overflow: Reject write

Communication failed: write refused

Error handling: prompt specific error message

D. Any method (such as equations, mathematical algorithms, logical operations) used to convert system inputs to corresponding outputs:

Direct output

E. Validation of the output data.

A. 输入数据的有效性检测。

B. 操作的确切次序:先由教师或助教发布公告或邮件,学生才可以查看。

C. 对异常情况的回应,例如:

	溢出:拒绝写入
	
	通信失败:拒绝写入
	
	错误处理:提示具体错误信息

D. 用于把系统输入转换到相应输出的任何方法(诸如方程式,数学算法,逻辑操作):
	
	直接输出
		
E.	对输出数据的有效性检测。
\subsubsection{输出}
A detailed description of all output data for this function, this description includes:

Where is the output (eg printer, file): client

Quantity: several

Time requirement: the semester is in progress

Handling illegal values: return error code

Error message: return error message

对该功能所有输出数据的详细描述,这个描述包括:

	输出的到何处(如打印机,文件):客户端

	数量:若干

	时间要求:学期进行中

	对非法值的处理:返回错误代码

	错误消息:返回错误信息
		
\subsection{R.INTF.CALC.006 成绩查询}
\subsubsection{介绍}

The results enquiry is completed by the students. The results enquiry can only be used to check the student's own performance and cannot be checked for other people's grades and average scores.

成绩查询由学生完成,成绩查询仅可查询学生本人的成绩不可查阅他人成绩和平均分等。

\subsubsection{输入}
A detailed description of all input data for this function, including:

Input source: grades entered by the teacher

Quantity: several

Time requirement: after the end of the semester

Valid input range with precision and tolerance: input should be an integer, 32-bit int

对该功能所有输入数据的详细描述,包括:

	输入来源:教师输入的成绩

	数量:若干
	
	时间要求:学期结束后
	
	包含精度和容忍度的有效输入范围:输入应为整数,为32位int型

\subsubsection{处理}

A. Validation of input data.

B. The exact sequence of operations: Students are entered into the system by the teacher before they can be queried.

C. Response to anomalies, such as:

Overflow: Reject write

Communication failed: write refused

Error handling: prompt specific error message

D. Any method (such as equations, mathematical algorithms, logical operations) used to convert system inputs to corresponding outputs. For example, this might describe the following:

Directly output the scores of students corresponding to the student number

E. Validation of the output data.

A. 输入数据的有效性检测。

B. 操作的确切次序:先由教师将学生成绩录入系统,学生才可以查询。

C. 对异常情况的回应,例如:

	溢出:拒绝写入
	
	通信失败:拒绝写入
	
	错误处理:提示具体错误信息

D. 用于把系统输入转换到相应输出的任何方法(诸如方程式,数学算法,逻辑操作)。例如,这可能描述下列方面:

	直接输出对应学号学生的成绩
		
E.	对输出数据的有效性检测。
\subsubsection{输出}
A detailed description of all output data for this function, this description includes:

Where is the output (eg printer, file): client

Quantity: several

Time requirement: after the teacher enters the grade

Effective output range with precision and tolerance: integer, 32-bit int

Handling illegal values: return error code

Error message: return error message

对该功能所有输出数据的详细描述,这个描述包括:

	输出的到何处(如打印机,文件):客户端

	数量:若干

	时间要求:教师录入成绩后

	包含精确度和容忍度的有效输出范围:整数,32位int型

	对非法值的处理:返回错误代码

	错误消息:返回错误信息
		
\subsection{R.INTF.CALC.007 评课系统}
\subsubsection{介绍}

The review community is reviewed by students at the end of each semester and can be viewed at any time.

评课社区由学生在每学期结束后发表评论,任何时间都可以查阅

\subsubsection{输入}

A detailed description of all input data for this function, including:

Input source: Student evaluation

Quantity: several

Time requirement: after the end of the semester

对该功能所有输入数据的详细描述,包括:

	输入来源:学生评价

	数量:若干

	时间要求:学期结束后
		
\subsubsection{处理}
A. Validation of input data.

B. The exact sequence of operations: Students can evaluate at the end of the semester and can be consulted at any time.

C. Response to anomalies, such as:

Overflow: Reject write

Communication failed: write refused

Error handling: prompt specific error message

D. Any method (such as equations, mathematical algorithms, logical operations) used to convert system inputs to corresponding outputs. For example, this might describe the following:

Direct output to the client

E. Validation of the output data.

A. 输入数据的有效性检测。

B. 操作的确切次序:学生在该学期课程结束后可以评价,任何时间都可以查阅。

C. 对异常情况的回应,例如:

	溢出:拒绝写入
	
	通信失败:拒绝写入
	
	错误处理:提示具体错误信息

D. 用于把系统输入转换到相应输出的任何方法(诸如方程式,数学算法,逻辑操作)。例如,这可能描述下列方面:

	直接输出到客户端

E.	对输出数据的有效性检测。
\subsubsection{输出}

A detailed description of all output data for this function, this description includes:

Where is the output (eg printer, file): client

Quantity: several

Time requirement: after the end of the semester

Handling illegal values: return error code

Error message: return error message

对该功能所有输出数据的详细描述,这个描述包括:
	
	输出的到何处(如打印机,文件):客户端
	
	数量:若干
	
	时间要求:学期结束后
	
	对非法值的处理:返回错误代码
	
	错误消息:返回错误信息

\section{性能需求}

For the integrated educational administration system, the amount of data is large, and there are many data processed during the peak period. Need to develop appropriate query and insertion algorithms and guarantee timing issues.

对于综合教务系统,数据量大,在高峰期运算处理数据、访问量多。需要开发合适的查询和插入算法,并且保证时序问题。

\subsection{性能需求1}
This subsection should describe the static and dynamic quantitative requirements for software (or human and software interaction) as a whole.

Static quantitative requirements may include:

A. Number of supported terminals: no less than 5000

B. Number of users supported for simultaneous use: no less than 5000

C. Number of documents and records processed: no less than 10,000

D. Table and file size: no more than 64M

Dynamic quantitative requirements may include:

A. Specific time period under normal and peak workload conditions: one hour

B. The number of transactions and tasks processed and the amount of data: no less than 5000

本子章节应从整体上描述静态和动态的量化的对软件(或人与软件交互)的需求。

静态的量化需求可能包括:

A. 支持的终端数目:不低于5000

B. 支持的同时使用的用户数目:不低于5000

C. 处理的文件和记录的数目:不低于10000

D. 表和文件的大小:不大于64M

动态的量化需求可能包括:

A. 在正常和峰值工作量条件下特定时间段:一小时

B. 处理的事务和任务的数目以及数据量:不低于5000

\section{外部接口需求}
\subsection{用户接口}
<The interface of the system with the User and vice versa should be explained in detail. >

详细描述系统与用户之间的接口

The features that the software must support for each human machine interface. For example, if a system user operates through a display terminal, it should include the following:

Required screen format: computer screen

Page planning and report or menu content: should include all of the above features

Input and output related timing: first input and then output

对每种人机界面,软件所必须支持的特性。例如,如果系统用户通过一个显示终端进行操作,那么应包含下述内容:

要求的屏幕格式:电脑端屏幕

页面规划及报告或菜单的内容:应包含上述全部功能

输入和输出的相关时序:先输入再输出


\subsection{软件接口}
<The interface with other system/modules/projects should be explained in detail. >

详细描述与其他系统 /模块 /项目之间的接口

Describes how to use the other (required) software products. (such as data management system, operation system, or algorithm tools package), and the interfaces to other application systems (such as interfaces between the protocol process system and the database management system )
For each required software product, following information should be provided:
A. Name
B. Mnemonic symbol
C. Version number
D. Source

在此应描述如何使用其它(必需的)软件产品(例如,数据管理系统,操作系统,或算法工具包),以及与其它应用系统的接口(例如,协议处理系统和数据库管理系统之间的接口)。

对每个必需的软件产品,应提供下列信息:
A.	名字
B.	助记符
C.	版本号
D.	来源

\subsection{硬件接口}
<The interface with other hardware components should be explained in detail. >

详细描述与硬件的接口

Describes the logical features of the interface between the software and hardware components, including the equipment supported and how the equipment and protocol is supported. 

Defines the interfaces according to the content and format of the software/hardware protocol. If the interfaces have been clearly described in other documents, it is not necessary to describe in detail here. But the reference of those documents should be given.

在此描述软件产品和系统硬件组件之间接口的逻辑特征,也包括支持哪些设备、怎样支持这些设备和协议等。
 
按软/硬件协议内容和格式定义接口

\subsection{通讯接口}
<This should specify the various interfaces to communications such as local network protocols, etc.>

详细描述通讯接口,如本地网络协议等。

Defines the interfaces according to the content and format of the message/function. If the interfaces have been clearly described in other documents, it is not necessary to describe in detail here. But the reference of those documents should be given.

按消息/函数内容和格式定义接口。
